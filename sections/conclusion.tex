%% LaTeX2e class for student theses
%% sections/conclusion.tex
%% 
%% Karlsruhe Institute of Technology
%% Institute for Program Structures and Data Organization
%% Chair for Software Design and Quality (SDQ)
%%
%% Dr.-Ing. Erik Burger
%% burger@kit.edu
%%
%% Version 1.3, 2016-12-29

\chapter{Conclusion and Future Work}
\label{ch:Conclusion}

The main contribution of this work was to introduce a theoretically motivated \emph{semantic-distributional word distance measure} that allows to improve 
short text classification with small datasets. By using a state-of-the-art clustering algorithm, we showed that we can perform 
lexical substiution on unfrequent terms and OOV-words and, hence, make the classification process more robust towards noise and
unknown vocabulary. Moreover, as demonstrated in Section \ref{ssec:importance-word-frequencies}, we illustrated that employing semantical
similarity only might be not beneficial for certain classification tasks. Instead, we propose to include supervised, distributional information yielded 
by the word statistics in the underlying training dataset. As opposed to methods proposed in the past, that exclusively used either distributional or semantic
similarity to perform term clustering \cite{baker1998distributional, ma2015using}, our method achieved improvements in TC in all the evaluated datasets.
Moreover, since our method is simply used as a preprocessing procedure to textual data, it doesn't impose any limitation on the choice 
of the classification algorithm. We also identified some open ends that need to be further researched. 
E.g., as mentioned in Section \ref{ssec:adding-linear-correction-term}, an interesting objective for further work, would be to clarify how to 
incorporate semantic similarity yielded by word embeddings in the Bayesian priors without the need of a linear correction term and 
the additional parameter $\beta$. Also, in the preprocessing method proposed in \ref{alg:preprocess-training-data}, the term clustering 
algorithm is applied to each size of $N$-grams separately. However, it might be more useful to have the possibility to compare and cluster on phrases of variable length, such as  e.g. 'not bad' and 'good' \cite{liu2015learning}. 
Finally, it would be interesting to address the question of whether 
the good results in short text classification achieved with our method
could be our transferred larger scale datasets, by employing semantic-distribuational term clustering in combination with more complex linguistic features, 
such as larger $N$-Grams, PoS-Tags or dependency based features \cite{komninos2016dependency}. 