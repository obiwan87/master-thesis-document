%% LaTeX2e class for student theses
%% sections/abstract_de.tex
%% 
%% Karlsruhe Institute of Technology
%% Institute for Program Structures and Data Organization
%% Chair for Software Design and Quality (SDQ)
%%
%% Dr.-Ing. Erik Burger
%% burger@kit.edu
%%
%% Version 1.3, 2016-12-29

\Abstract

Als Textklassifikation (TK) bezeichnet man die Problemstellung, textuelle Daten vordefinierten Klassen oder Kategorien zuzuordnen. 
TK ist nicht nur Gegenstand vieler wissenschaftlicher Arbeiten, sondern auch von immer größerer praktischer Bedeutung 
für Industrie und Wirtschaft. Die Prädiktionsfähigkeit dieser Systeme beruht hauptsächlich auf der Qualität und vor allem aber auch 
auf der Quantität der Daten, mit denen diese trainiert werden. In vielen Fällen müssen diese Trainingsdaten manuell erzeugt, bzw.
gekennzeichnet werden, was häufig einen beträchtlichen zeitlichen und finanziellen Aufwand bedeutet.  
Oft ist es nicht praktikabel, die eigentlich optimale Anzahl an Trainingsdaten manuell zu erzeugen und es muss
auf einen kleineren Trainingsdatensatz zurückgegriffen werden. In dieser Arbeit wird eine Methode vorgestellt, 
welche die Klassifizierungsgenauigkeit gerade in solchen Szenarien verbessern kann. 
Erreicht wird dies durch eine neue \emph{semantisch-statistische Wort-Distanzmetrik}.
In Kombination mit Clustering-Algorithmen kann diese Metrik verwendet werden, um den Merkmalsraum unter 
Beibehaltung der Semantik zu reduzieren.  Eine besondere Eigenschaft des neuen Distanzmaßes ist die Kombination semantischer 
Information aus Word Embeddings \cite{mikolov2013distributed} mit intrinsischen, statistischen Eigenschaften der Trainingsdaten. Die in dieser Arbeit vorgestellte 
 Methode wurde mit State-of-the-Art Klassifikatoren wie Multinomial Naive Bayes und Support Vektor Maschinen evaluiert. Im Gegensatz zu 
 existierenden Ansätzen (Wort-Clustering \cite{baker1998distributional, ma2015using}) konnten hier Verbesserungen in allen getesteten Datensätzen erzielt werden.
